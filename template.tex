\documentclass[12pt]{article}
\usepackage[a4paper,margin=1in]{geometry}
\usepackage{setspace}
\usepackage{titling}

\usepackage{amsmath}
\usepackage{amssymb}
\usepackage{tikz}
\usepackage{booktabs}

\begin{document}

\begin{titlepage}
    \begin{center}
        \vspace*{3cm}

        \textbf{\Large From Combinatorics to Calculus}\\[2.5cm]

        \textbf{\Large YOUR TITLE}\\[2.5cm]

        \textbf{\large Student name}\\[0.5cm]
        \textbf{XXX}\\[2cm]

        \textbf{\large Major}\\[0.5cm]
        \textbf{XXX}\\[2cm]

        \textbf{\large Supervisor name}\\[0.5cm]
        \textbf{Dan Ciubotaru}\\[3cm]
    \end{center}
\end{titlepage}

\section{Research Problem and Objectives}
\subsection{Research Problems and Background}

Research Background: This section is dedicated to the research problem and background of your study. You can provide an overview of the research topic, its significance, and the motivation behind your research.

\subsection{Research Objectives}

The research aims to ...

The primary objectives of this research are:

\begin{itemize}
    \item A
    \item B
\end{itemize}

\section{Literature Review}

Review of previous work: This section is dedicated to the literature review of your research topic. You can discuss the existing research in the field and highlight the gaps in the literature that your research aims to address.

For example:
The study of harmonic maps and spinors has been extensively explored in the literature. Eells and Lemaire \cite{eells1988} provided a comprehensive report on harmonic maps, detailing significant advancements and methodologies in the field. Hitchin \cite{hitchin1974} made notable contributions with his work on harmonic spinors, which has been foundational for subsequent research. Bär \cite{bar1996, bar1998} further expanded on this by examining metrics with harmonic spinors and discussing their implications in gauge theories of gravitation. These works collectively highlight the depth and breadth of research in harmonic maps and spinors, offering a robust framework for further exploration and application in various mathematical and physical contexts.

\section{Main Results}

Main Body: This section is dedicated to your Research Methodology, where you can delve into the specifics and illustrate with examples how you addressed certain challenges through rigorous proof or computation.

\section{Conclusion and Discussion}

Expected Results and Significance: This section is dedicated to the expected results of your research and the significance of your findings. You can also discuss the potential impact of your research on the field and its applications.

\begin{thebibliography}{9}

\bibitem{eells1988}
J. Eells, L. Lemaire, Another report on harmonic maps, Bull. London Math. Soc. 20 (5) (1988) 385–524. MR0956352 (89i:58027). Two reports on harmonic maps, 69–208, World Sci. Publ., River Edge, NJ, 1995.

\bibitem{hitchin1974}
N. Hitchin, Harmonic spinors, Adv. Math. 14 (1974) 1–55.

\bibitem{bar1996}
C. Bär, Metrics with harmonic spinors, Geom. Funct. Anal. 6 (6) (1996) 899–942.

\bibitem{bar1998}
C. Bär, On harmonic spinors. Gauge theories of gravitation (Jadwisin, 1997), Acta Phys. Polon. B 29 (4) (1998) 859–869.

\end{thebibliography}

\end{document}
